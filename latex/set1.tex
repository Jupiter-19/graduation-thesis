% !TEX program = xelatex


%论文分类号
\renewcommand{\zjuclass}{TM888} 

%论文密级
\renewcommand{\zjusecurity}{}

\renewcommand{\zjutitlec}{带守恒量的自治随机微分方程}%中文论文题目
\renewcommand{\zjutitlecb}{的数值解研究}%中文论文题目第二行,若无请留空
\renewcommand{\zjutitlecc}{}%中文论文题目第三行,若无请留空
\renewcommand{\zjutitlee}{Numerical Study of Stochastic Autonomous}%英文论文题目
\renewcommand{\zjutitleeb}{Differential Equations with Conserved Quantities}%英文论文题目第二行,若无请留空
\renewcommand{\zjutitleec}{}%英文论文题目第三行,若无请留空


%作者姓名
\renewcommand{\zjuauthor}{张少杰} 

%作者学号
\renewcommand{\zjuauthorid}{21835030}

%指导教师 
\renewcommand{\zjumentor}{程晓良 教授}

%合作导师(如果有的话请取消注释)
%\renewcommand{\zjumentorco}{}

%专业名称
\renewcommand{\zjumajor}{计算数学}

%研究方向
\renewcommand{\zjusubject}{随机微分方程数值解}

%所在学院
\renewcommand{\zjuschool}{数学科学学院}

%提交日期
\renewcommand{\zjuapprovaldate}{二〇二一年一月}

%答辩日期
\renewcommand{\zjudefencedatec}{} % 二〇二一年三月}
\renewcommand{\zjudefencedatee}{} % March 2020}

%论文评阅人(格式:姓名 ~ 职称 ~ 单位)
\renewcommand{\zjurevieweronec}{}
\renewcommand{\zjurevieweronee}{}

\renewcommand{\zjureviewertwoc}{}
\renewcommand{\zjureviewertwoe}{}

\renewcommand{\zjureviewerthreec}{}
\renewcommand{\zjureviewerthreee}{}

\renewcommand{\zjureviewerfourc}{}
\renewcommand{\zjureviewerfoure}{}

\renewcommand{\zjureviewerfivec}{}
\renewcommand{\zjureviewerfivee}{}

%答辩委员会(姓名\职称\单位)
\renewcommand{\zjucommitteemainc}{}
\renewcommand{\zjucommitteemaine}{}

\renewcommand{\zjucommitteeonec}{}
\renewcommand{\zjucommitteeonee}{}

\renewcommand{\zjucommitteetwoc}{}
\renewcommand{\zjucommitteetwoe}{}

\renewcommand{\zjucommitteethreec}{}
\renewcommand{\zjucommitteethreee}{}

\renewcommand{\zjucommitteefourc}{}
\renewcommand{\zjucommitteefoure}{}

\renewcommand{\zjucommitteefivec}{}
\renewcommand{\zjucommitteefivee}{}

%\renewcommand{\zjurevieweronec}{陈某某 ~ 教授 ~ XX大学}
%\renewcommand{\zjurevieweronee}{Moumou Chen ~Prof. ~XX University}

%\renewcommand{\zjureviewertwoc}{张某某 ~ 教授 ~ XX大学}
%\renewcommand{\zjureviewertwoe}{Moumou Zhang ~Prof. ~XX University}

%\renewcommand{\zjureviewerthreec}{李某某 ~ 教授 ~ XX大学}
%\renewcommand{\zjureviewerthreee}{Moumou Li ~Prof. ~XX University }

%\renewcommand{\zjureviewerfourc}{吴某某 ~ 教授 ~ XX大学}
%\renewcommand{\zjureviewerfoure}{Moumou Wu ~Prof. ~XX University}

%\renewcommand{\zjureviewerfivec}{杨某某 ~ 教授 ~ XX大学}
%\renewcommand{\zjureviewerfivee}{Moumou Yang ~Prof. ~XX University}

%答辩委员会(姓名\职称\单位)
%\renewcommand{\zjucommitteemainc}{郑某某 ~ 教授 ~ 浙江大学}
%\renewcommand{\zjucommitteemaine}{Moumou Zheng~ Prof. ~Zhejiang University}

%\renewcommand{\zjucommitteeonec}{郑某某 ~ 教授 ~ 浙江大学}
%\renewcommand{\zjucommitteeonee}{Moumou Zheng ~Prof. ~Zhejiang University}

%\renewcommand{\zjucommitteetwoc}{刘某某 ~ 教授 ~ 浙江大学}
%\renewcommand{\zjucommitteetwoe}{Moumou Liu ~Prof. ~Zhejiang University}

%\renewcommand{\zjucommitteethreec}{程某某 ~ 教授 ~ 浙江大学}
%\renewcommand{\zjucommitteethreee}{Moumou Chen ~Prof. ~Zhejiang University}

%\renewcommand{\zjucommitteefourc}{杨某某 ~ 教授 ~ 浙江大学}
%\renewcommand{\zjucommitteefoure}{Moumou Yang ~Prof. ~Zhejiang University}

%\renewcommand{\zjucommitteefivec}{吴某某 ~ 教授 ~ 浙江大学}
%\renewcommand{\zjucommitteefivee}{Moumou Wu ~Prof. ~Zhejiang University}



% 添加代码格式

% 代码框(matlab)
\lstdefinestyle{styleM}{
	language=Matlab,
	numbers=left, %设置行号位置
	numberstyle=\tiny, %设置行号大小
	keywordstyle=\color{blue}, %设置关键字颜色
	commentstyle=\color[cmyk]{1,0,1,0}, % 设置注释颜色
	frame=single, %设置边框格式
	escapeinside=``, %逃逸字符(1左面的键),用于显示中文
	breaklines, %自动折行
	extendedchars=false, %解决代码跨页时,章节标题,页眉等汉字不显示的问题
	xleftmargin=2em,xrightmargin=2em, aboveskip=1em, % 设置边距
	tabsize=4, %设置tab空格数
	showspaces=false %不显示空格
	numberstyle=\small
}

%代码框(python)
\lstdefinestyle{styleP}{
	language=Python,
	numbers=left, %设置行号位置
	numberstyle=\tiny, %设置行号大小
	keywordstyle=\color{blue}, %设置关键字颜色
	commentstyle=\color[cmyk]{1,0,1,0}, % 设置注释颜色
	frame=single, %设置边框格式
	escapeinside=``, %逃逸字符(1左面的键),用于显示中文
	breaklines, %自动折行
	extendedchars=false, %解决代码跨页时,章节标题,页眉等汉字不显示的问题
	xleftmargin=2em,xrightmargin=2em, aboveskip=1em, % 设置边距
	tabsize=4, %设置tab空格数
	showspaces=false %不显示空格
	numberstyle=\small
}

%代码框(R)
\lstdefinestyle{styleR}{
	language=R,
	numbers=left, %设置行号位置
	numberstyle=\tiny, %设置行号大小
	keywordstyle=\color{blue}, %设置关键字颜色
	commentstyle=\color[cmyk]{1,0,1,0}, %设置注释颜色
	frame=single, %设置边框格式
	escapeinside=``, %逃逸字符(1左面的键),用于显示中文
	%breaklines, %自动折行
	extendedchars=false, %解决代码跨页时,章节标题,页眉等汉字不显示的问题
	xleftmargin=2em,xrightmargin=2em, aboveskip=1em, %设置边距
	tabsize=4, %设置tab空格数
	showspaces=false %不显示空格
}
