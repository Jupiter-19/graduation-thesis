% !TeX encoding = UTF-8
\chapter{总结与展望}\label{chap7}


本文探究自治随机偏微分方程的数值解,由于其同时具有偏微分方程和随机过程的属性,分别从单一样本轨道的数值解和整体的概率解两个维度进行分析. 
对于时间 $T$ 较小的情况,本文综述经典的数值算法:Euler-Maruyama 格式、Milstein 格式、Taylor格式等. 并对带守恒量的问题介绍了维持守恒量的数值格式和投射算法. 
对于时间 $T$ 较大的情况,本文介绍其具有稳定概率分布的条件. 并分析当自治随机偏微分方程存在守恒量时,稳定解必然不唯一. 
本文的创新点在于探究带守恒量的问题时,对于单一样本轨道的数值解提出新的投射方法,对于整体的概率解,提出了稳定解不唯一并给出了能收敛到初值对应的稳定解的加速算法. 

通过实验,本文发现当状态“连通”且“连通”的状态的总测度有限时,SDE 总是存在稳定解. 带守恒量的问题稳定解不唯一是因为守恒量破坏了各状态之间的“连通性”,但本文没有分析探究这种“连通性”. 
另外本文的随机微分方程仅考虑了布朗运动的随机项 $B_t$,而更一般的分数阶布朗运动的随机项则未进行理论分析和相关实验. 可以推测的是,一般的分数阶布朗运动由于“时间记忆”的特点,即使存在稳定解,收敛速度也会更慢. 
这些有趣的角度非常值得进行进一步进行探究. 




% 发散:初值的扰动与混沌 
% 尝试 Fokker-Planck 方程的求解
