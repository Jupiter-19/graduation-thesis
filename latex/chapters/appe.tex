% !TeX encoding = UTF-8
\chapter{附录}
\section*{部分实验代码}
\subsection*{3.2节 Euler格式的python代码}

\scriptsize{
\begin{lstlisting}[style=styleP]
'''
d X = bX dt + sigma X dB
方程参数: X0,b,sigma
时间参数:T
格式参数:h
'''
import numpy as np
import matplotlib.pyplot as plt

class Euler():
	def __init__(self,X0,b,sigma,h,T):
		self.X0 = X0
		self.b,self.sigma = b,sigma
		self.h,self.T = h,T
		self.set_para()

	def set_para(self):
		self.N = int(self.T / self.h)
		self.sqrt_h = np.sqrt(self.h)
		self.t = [self.h * i for i in range(self.N+1)]
	
	def sample(self , myplot = 0):
		self.Brown = np.random.normal(0 , 1, self.N) * self.sqrt_h
		X = [self.X0 for i in range(self.N+1)]
		
		for i in range(self.N):
			X[i+1] = X[i] + self.b * X[i] * self.h + self.sigma * X[i] * self.Brown[i]
		
		if myplot:
			plt.figure(figsize=(20,10))
			plt.plot(self.t,X)
		
		real = self.X0 * np.exp((self.b-self.sigma**2/2)*self.T+self.sigma*np.sum(self.Brown))
		return (X[-1] , real)
	
	def test(self):
		EX = self.X0 * np.exp(self.b * self.T)
		DX = (self.X0**2 * np.exp(2*self.b*self.T) * (np.exp(self.sigma**2*self.T)-1))

		result,error = [],[]
		for i in range(10000):
			calc,real = self.sample()
			result.append(calc)
			error.append(real-calc)
		
		m_mean,m_std = np.mean(result),np.std(result) 
		print(EX , DX , '\n' , m_mean , m_std ** 2)
		print(np.mean(error) , np.std(error)**2)
		
		result = [ x for x in result if x > m_mean-5*m_std and x < m_mean + 5*m_std]
		plt.hist(result,bins = 100)

method = Euler(X0 = 2 , b = -2, sigma = 2 , h = 1e-4 , T = 5)
method.sample(myplot = 1)
method.test()
\end{lstlisting}
}



\subsection*{4.2节代码}
\scriptsize{
\begin{lstlisting}[style=styleP]
'''
方程:dX = bx dt + sqrt(2)*(1-x**2) dB
固定参数:b
可变参数:T
精度参数:h,格式
'''
import numpy as np
import matplotlib.pyplot as plt

class Stable():
	def __init__(self,b,h,T):
		self.b = b
		self.h = h
		self.T = T
		self.set_para()

	def set_para(self):
		self.N = int(self.T / self.h)
		self.sqrt_h = np.sqrt(self.h)
		self.sqrt_2 = np.sqrt(2)
		self.t = [self.h * i for i in range(self.N+1)]

	def sample(self,myplot = 0):
		self.Brown = np.random.normal(0 , 1, self.N) * self.sqrt_h
		X = [0 for i in range(self.N+1)]
		
		tmp = np.random.normal(0,0.2)
		while np.abs(tmp) >= 1:
			tmp = np.random.normal(0,0.2)
		X[0] = tmp        
			
		
		for i in range(self.N):
		# Milstein 格式
		X[i+1] = X[i] + self.b * X[i] * self.h + self.sqrt_2 * (1-X[i]**2) * self.Brown[i]
		X[i+1]+= (self.Brown[i]**2 - self.h) * 2 * X[i] * (X[i]**2-1)
		
		if X[i+1] > 1: X[i+1] = (X[i]+1) / 2
		if X[i+1] <-1: X[i+1] = (X[i]-1) / 2
		
		if myplot:
			plt.figure(figsize=(20,10))
			plt.plot(self.t,X)
		
		return X[-1]

	def measure(self):
		result = []
		for idx in range(500000):
			if idx % 20000 == 0: print(idx//5000+4)
			result.append(self.sample())
			
		plt.figure(figsize=(18,9))
		plt.xlim(-1,1)
		plt.hist(result , bins = 250)
		plt.show()

m = Stable(b = -2 , h = T/1000 , T = 1.0)
#m.sample(myplot=1)
m.measure()
\end{lstlisting}
}






\newpage

\subsection*{TaylorP格式下Kubo-oscillator 方程稳定解算法}
\scriptsize{
\begin{lstlisting}[style=styleP]
import os 
import numpy as np
import matplotlib.pyplot as plt
import pandas as pd

class Kubo():
	def __init__(self):
		self.b, self.sigma = 2,2
		self.h = 1e-2
		self.T = 80
	
	# 数据分布演化 num 次 
	def scheme(self,oldx,oldy,num):
		length = len(oldx)
		b,sigma,h = self.b , self.sigma, self.h
	
		newx,newy = [],[]
		for k in range(length):
			# 第 k 个点的模拟
			tx = [oldx[k] for _ in range(num+1)]
			ty = [oldy[k] for _ in range(num+1)]
			Brown = np.random.normal(0 , 1, self.N) * np.sqrt(self.h)
			for i in range(num):
				# Taylor格式
				tx[i+1] = tx[i] - b*ty[i]*h - sigma*ty[i]*Brown[i] - sigma**2*tx[i]/2*Brown[i]**2
				tx[i+1] +=(Brown[i]**3-3*h*Brown[i])/6 * sigma**3*ty[i]
				tx[i+1] +=(sigma**4/4*tx[i]-sigma**2*b*ty[i]-b**2*tx[i])/2*h*h
				tx[i+1] -= Brown[i]*h*(sigma**3/2*ty[i]+b*sigma*tx[i])
				ty[i+1] = ty[i] + b*tx[i]*h + sigma*tx[i]*Brown[i] - sigma**2*ty[i]/2*Brown[i]**2
				ty[i+1] -=(Brown[i]**3-3*h*Brown[i])/6 * sigma**3*tx[i]
				ty[i+1] +=(sigma**4/4*ty[i]+sigma**2*b*tx[i]-b**2*ty[i])/2*h*h
				ty[i+1] += Brown[i]*h*(sigma**3/2*tx[i]-b*sigma*ty[i])
				# 投射操作
				tmp = np.sqrt(tx[i+1]**2 + ty[i+1]**2)
				tx[i+1],ty[i+1] = tx[i+1]/tmp,ty[i+1]/tmp

			newx.append(tx[-1])
			newy.append(ty[-1])
		
		return newx,newy
	
	def test(self):
		num = int(0.2 / self.h)
		
		for n in range(1,int(self.T/0.2)+1):
			time = str(n*2//10).zfill(2)+'.'+str(n*2%10)
			print(time)
			if 'T='+time+'.csv' in os.listdir('D:\data\Kubo\point'):
				data = pd.read_csv('D:\data\Kubo\point\T='+time+'.csv')
				x = list(data['x'])
				y = list(data['y'])
				continue
		
			x,y = self.scheme(x,y,num)
			out = pd.DataFrame({'x':x , 'y':y})
			out.to_csv('D:\data\Kubo\point\T='+time+'.csv')
			
			plt.hist(x,bins = 100)
			plt.savefig('D:\data\Kubo\histX\T='+time+'.png')
			plt.show()
			
			plt.hist(y,bins = 100)
			plt.savefig('D:\data\Kubo\histY\T='+time+'.png')
			plt.show()
			
			plt.figure(figsize=(8,8))
			plt.scatter(x[:100],y[:100])
			plt.show()

k = Kubo()
k.test()
\end{lstlisting}
}







\section*{\ito 积分、Stratonovich积分下的二次变差项}
随机积分中,积分表达式 $\displaystyle \int B_t \md B_t$ 会多次出现. 同时注意到,本文的理论与数值格式中,$\frac12 \Delta B_t^2$ 这类二次变差项也会反复出现. 且在\ito 积分、Stratonovich积分下正负号正好相反. 此处简要计算说明. 
\[
\begin{aligned} 
S_{n}&=\sum_{i=0}^{n-1} B\left(t_{i}^{n}\right)\left[B\left(t_{i+1}^{n}\right)-B\left(t_{i}^{n}\right)\right] \\
&=
\sum_{i=0}^{n-1} \frac{B\left(t_{i+1}^{n}\right)+B\left(t_{i}^{n}\right)}{2}\left[B\left(t_{i+1}^{n}\right)-B\left(t_{i}^{n}\right)\right] 
-\sum_{i=0}^{n-1} \frac{B\left(t_{i+1}^{n}\right)-B\left(t_{i}^{n}\right)}{2}\left[B\left(t_{i+1}^{n}\right)-B\left(t_{i}^{n}\right)\right]\\
&=\frac12 B(T)^2 - \frac12 \sum_{i=0}^{n-1}\left[B\left(t_{i+1}^{n}\right)-B\left(t_{i}^{n}\right)\right]^{2} 
\longrightarrow \frac12 B(T)^2 - \frac12 T.\\
S^{\prime}_n&=\sum_{i=0}^{n-1} B\left(t_{i+1}^{n}\right)\left[B\left(t_{i+1}^{n}\right)-B\left(t_{i}^{n}\right)\right]
\longrightarrow \frac12 B(T)^2 + \frac12 T.\\
S_n''&=\sum_{i=0}^{n-1} \frac{B\left(t_{i+1}^{n}\right)+B\left(t_{i}^{n}\right)}{2}\left[B\left(t_{i+1}^{n}\right)-B\left(t_{i}^{n}\right)\right] = \frac12 B(T)^2
\end{aligned}
\]
第一种情况对应 \ito 积分意义,第二中情况对应 Stratonovich 积分意义. 


\iffalse


\begin{shaded}
	积分表计算:文献\cite{integer}
\end{shaded}


\section*{积分表}
下面的积分表用于数值积分的推导
\begin{table}[!htbp]
\caption{随机积分表}
\centering
\begin{tabular}{c|c||c|c}
	\ito 积分表达式 & 值 & Stratonovich 积分 & 值 \\
	\hline
	\rule{0pt}{15pt}$\int_t^{t+h} \md B_s$  & $\Delta B$  & 
		$\int_t^{t+h} \circ \md B_s$  & $\Delta B$   \\
	\rule{0pt}{15pt}$\int_t^{t+h} \int_t^s \md B_\theta \md s $ & $\Delta Z$ & 
		$\int_t^{t+h} \int_t^s \circ \md B_\theta \md s $ & $\Delta Z$ \\
	\rule{0pt}{15pt} $\int_t^{t+h}\int_t^s \md \theta \md B_s $ & $\Delta B \cdot h - \Delta Z$ &
		$\int_t^{t+h}\int_t^s \md \theta \circ\md B_s $ & $\Delta B \cdot h - \Delta Z$ \\
	\rule{0pt}{15pt} $\int_t^{t+h} \int_t^s \md B^1_\theta \md B^2_s$
\end{tabular}
\end{table}

\fi






