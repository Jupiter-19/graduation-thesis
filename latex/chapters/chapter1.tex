% !TeX encoding = UTF-8

\chapter{引言}
\label{chap_int}
\section{选题背景与意义}
随机过程是用来描述随机事件随时间演化的数学模型. 离散的演化模型,如时间序列模型,在各种实际问题中已经得到了广泛的应用. 而连续的演化模型,即本论文关注的随机偏微分方程,也越来越受到重视. 


对布朗运动的理解是随机微分方程的基础. 1827年,英国植物学家布朗(Robert Brown)注意到悬浮在水中的花粉颗粒呈现出很不规则的运动轨迹:急剧改变方向,无法预测. 
爱因斯坦在1905年发表的论文\cite{Einstein}被认为是理解布朗运动的重要进步. 论文研究布朗粒子在开始时位于原点,则经过时间 $t$ 后,粒子的分布服从高斯分布,其均值为0,方差为 $\sigma^2t$. 在概率论还不完备的年代,爱因斯坦以统计力学为基础的,引导得到热方程,然后用高斯分布来导出方程的解. 之后法国科学家 Perrin 通过实验证实了爱因斯坦的理论.


在对股票市场的数学模型的探索过程中,法国数学家 Louis Bachelier 使用步长取的非常小的随机游走分析,为之后的期权定价公式奠定了基础\cite{book_finance,book_fenxi}. 从数学的观点来看,随机游走的极限状态与布朗运动是等价的. 布朗运动具有 Markov 性质,即粒子在 $t$ 以后的位置与 $t$ 以前的位置信息是无关的,因而只要知道 $t$ 时刻的信息就可以了. 这与金融中的有效市场假说也是一致的. 


柯尔莫哥洛夫、维纳、伊藤清也都在布朗运动的研究中做出了巨大贡献. 柯尔莫哥洛夫发表于1931年的著作\cite{Kolmogoroff}将随机过程理论用数学分析表述,尤其是将随机系统与(常、偏)微分方程理论联系在一起,讨论了部分方程的解的存在性、唯一性、光滑性等问题. 维纳于1923年给出了布朗运动的严格数学定义和构造方式. 因此布朗运动也被称为维纳过程,并成为20世纪概率论的主要研究对象. 1942年,由布朗运动的样本轨迹处处连续但无处可微,伊藤清引入了对布朗运动和更一般的随机过程的新微分学,奠定了现代随机分析的基础,并成为了随机过程研究的强有力的有效手段之一\cite{ref_psg}. 


考虑在水中的布朗运动,如果水温上升,由于分子热运动运动加快,我们期望有速度更快的粒子做更多的碰撞,这点可以用增加扩散速度来体现.再考虑水中各处水温不同,这时需要在布朗运动上再叠加一个“漂移”项,用来表示的随机微分方程的记号是
\begin{equation}
	\md X_t =  b(X_t) \md t +  \sigma(X_t) \md B_t .
\end{equation}
它的一个重要例子是数理金融学中重要的 Black-Scholes 模型\cite{book_finance}. 在这个模型里,股票的价格满足一个上面类似的随机微分方程. 其中 $\sigma(x) = \sigma x$,$b(x)=bx$. 


更一般的物理问题中,系统内部存在难以精确测量的随机项或扰动,如文献\cite{paper_SDE_example_1}中模拟的动态流体系统,Stratonovich 意义下对该随机系统建模
\[
\md \left(\begin{array}{c}
r \\
v \\
\varepsilon
\end{array}\right)=\left(\begin{array}{c}
v \\
F(r)-v \\
v^{2}
\end{array}\right) \md t+\left(\begin{array}{c}
0 \\
(2 \alpha \varepsilon)^{\frac{1}{2}} \\
-v(2 \alpha \varepsilon)^{\frac{1}{2}}
\end{array}\right) \circ \md B_t
\]
$r,v,\varepsilon$ 分别表示位置、速度和能量. 而Stratonovich是与 \ito 相对应的随机积分方式.扰动项的存在使得问题的解表现为概率解. 同时,扰动项服从特定概率分布,使得问题建模相对于确定性微分方程更加精确. 


1947年,理查德·费曼和马克·卡茨将随机过程和偏微分方程结合在一起,提出了费曼-卡茨公式,将某些抛物型偏微分方程的解写成随机过程的条件期望,从而将求此类微分方程的数值解转化为模拟随机过程的路径. 反过来,此类随机过程的期望可以通过确定性的计算偏微分方程得到. 考虑偏微分方程
\[
\frac{\partial u}{\partial t}+\mu(x, t) \frac{\partial u}{\partial x}+\frac{1}{2} \sigma^{2}(x, t) \frac{\partial^{2} u}{\partial x^{2}}-V(x, t) u=f(x, t) , \quad 
u(x,T) = \psi(x).
%t \in[0, T], x \in \mathbb{R}
\]
这个偏微分方程的解函数可以写成下面随机过程的(条件)期望:
\[
u(x, t)=E\left[\int_{t}^{T} e^{-\int_{t}^{s} V\left(X_{\tau}\right), d \tau} f\left(X_{s}, s\right) d s+e^{-\int_{t}^{T} V\left(X_{\tau}\right) d \tau} \psi\left(X_{T}\right) \Bigl| X_{t}=x\right]
\]
其中 $X = (X_t ; t \ge 0)$ 是由以下的随机微分方程(stochastic differential equation, SDE)
\[
\md X_t = \mu(X_t,t) \md t + \sigma(X_t,t) \md B_t,
\]
决定的伊藤随机过程. 


本文所关心的随机微分方程正是将微分方程和随机过程这两种工具结合起来研究的模型. 除了高斯白噪声,我们还可以使用泊松白噪声模拟随机脉冲信号或跳过程,模型如下
\[
\left\{
\begin{aligned}
&\md X(t) = f(t,X(t^-),\omega) \md t + g(t,X(t^-),\omega) \md  W_t + \sigma (t,X(t^-),\omega) \md N_t ,\qquad 0<t\le T,\\
& X(0^-) = X_0.
\end{aligned}
\right.
\]
这里 $W_t$ 是广义的 Weiner 过程(包括分数阶Brown运动),$N_t$ 是 Poisson 过程. 可以参见2010年 Bruti-Liberati 和 Platen 关于金融领域中带跳随机常微分方程数值解的专著\cite{book_jump1}和文献综述\cite{book_jump2}.



随机微分方程的模型在金融和物理系统中起到十分重要的作用,与确定性的微分方程相比,随机微分方程可以提供更真实的模型并捕获基础系统的更多行为. 但由于难以获得随机微分方程的精确解,研究随机微分方程的数值方法就非常重要. Euler–Maruyama、Milstein、Runge-Kutta 等数值格式是求解随机微分方程的常用的近似解技术. 除了这些单步或多步方法,用于求解确定性微分方程的 Galerkin 方法也可以推广为处理随机系统的随机 Galerkin 方法. 使用带随机扰动的多项式基函数,称为 gPC 基函数,在合适的函数空间中寻找问题的解. 在文献\cite{book_buqueding}中,可以看到随机Galerkin方法以外的谱方法. 目前这方面的高效算法还很少,且往往只能作用一些特定的问题,同时数值算法的收敛性、收敛阶的论证也存在诸多困难. 本文旨在分析自治随机微分方程的短时间的数值解和长时间的稳定解. 












\section{文献综述}

在 SDE 的数值解的研究中,保留随机系统特征的数值算法的构造是研究的活跃领域. 
这类数值格式的例子有:
\begin{enumerate}
	\item[$\bullet$] 为 SDE 构建非负解的开创性工作\cite{zongshu5};
	\item[$\bullet$] 通过欧拉格式渐近保存诸如矩,分布和不变测度之类的量\cite{zongshu6}; 
	\item[$\bullet$] 通过中点格式保存线性SDE的一阶矩和二阶矩\cite{zongshu7};
	\item[$\bullet$] 具有加性和乘性噪声项的随机哈密顿系统的辛积分器\cite{zongshu16,zongshu17};
	\item[$\bullet$] 分离的 Euler–Maruyama 方法保持二阶矩线性增长的特性\cite{zongshu14};
	\item[$\bullet$] 具有守恒能量的随机哈密顿系统的能量保守差分方法\cite{zongshu18};
	\item[$\bullet$] SDE 的边界保留半解析数值算法\cite{zongshu21};
	\item[$\bullet$] 随机李群积分器\cite{zongshu12};
\end{enumerate}

本文关注于带单一或多个守恒量自治随机微分方程的数值解法. 具体包括每个样本轨道上的解和整体的概率解. 
由于守恒量的物理属性,例如在力学和天文学中,守恒量往往表达某种能量或物理学定律中的不变量. 对于确定性微分方程,保留守恒量的数值方法对于执行可靠的数值计算和具有较长的时间稳定性很重要\cite{zongshu4}. 
自然地,对于随机微分方程,如果系统具有守恒量,那么构造保留此类守恒量的数值方法就很有意义. 

实际上,在文献中已经有一些关于 SDE 守恒量的概念,例如\cite{zongshu19,zongshu25}. 但关于这个问题的著作很少. 先前的一项工作\cite{zongshu18}提出了随机哈密顿系统的能量守恒差分方案,并提出了该方案的局部误差阶. 另一著作\cite{zongshu20}仅考察了其中提出的合成方法的一个优点.	













\section{文章结构} 


本文首先考虑的问题是如下的自治随机常微分方程(SODEs):
\begin{equation}\label{eq1.3.1}
\left\{
	\begin{aligned}
	& \md X(t) = b(X(t)) \md t + \sigma(X(t)) \md B_t,\qquad  0<t \le T,\\
	& X(0) = X_0 \in \R^d.
	\end{aligned}
\right.
\end{equation}
这里 $b:\R^d \to \R^d$ 和 $\sigma=(\sigma^1,\sigma^2,\cdots,\sigma^m):\R^d\to\R^{d \times m}$ 是给定的函数,$B_t$是$m$维标准Brown运动(也称 Wiener 过程). 该问题解释为 \ito 意义下的随机积分方程
\begin{equation}\label{eq1.3.2}
	X_t = X_0 + \int_0^t b(X_s) \md s + \int_0^t \sigma(X(s)) \md B_s,\qquad t\in[0,T]
\end{equation}


本文的第二章介绍随机分析和随机微分方程中的基本但必要的知识,这辅助我们去理解微分形式的问题\ref{eq1.3.1}和积分形式的问题(\ref{eq1.3.2})及其解在概率下的意义.

在求解传统的常、偏微分方程的数值解的过程中,我们使用的是有限差分、有限元方法等工具. 第三章介绍这些工具如何用于随机微分方程,本文分别介绍了单步法、多步法和谱方法中的经典结果,并给出收敛性结果. 值得注意的是,目前很多数值方法的收敛性结果只适用于自治随机偏微分方程.

第四章关心的是解的长时间性质. 注意到问题(\ref{eq1.3.1})是时间齐次的连续Markov过程,即确定 $X$ 在 $t$ 时的状态,则 $t$ 以后的状态只与 $X_t$ 和函数 $b,\sigma$ 有关,而与时间 $t$ 无关. 那么可以很自然地猜测,当时间 $t$ 趋向无穷时,$X_t$ 在概率意义下可能会收敛到某一个稳定的解 $X^*$. 在第四章中,本文介绍稳定解的算子方式的定义及方程存在稳定解的一个充分但不必要条件,并说明在该条件下,稳定解是唯一的,且 $X^*$ 就是方程的极限概率解 $X_\infty \triangleq \displaystyle \lim_{t\to\infty}X_t$. 同时,本文还给出了数值求解稳定解的加速算法. 

第五章在随机微分方程上添加守恒量的信息. 守恒量要求方程的解 $X_t$ 的样本轨道始终在 $\R^{d+1}$ 的某个流形上. 对此,本文介绍两种数值方法,使得数值格式在维持精度的同时能保证守恒量是不变的. 对于含有守恒量的 SDE 方程,本文探讨了若其存在稳定解,则稳定解可能不唯一. 并结合保守恒量算法,本文给出了守恒量不唯一时,稳定解的加速算法. 

第六章设计数值实验,验证本文的算法的数值特性:收敛精度、守恒量不变、收敛到稳定解. 数值实验分别考虑了 $T$ 较小时单条样本轨道的精度和 $T$ 非常大的稳定解的形态. 

最后一章简要总结了本文结果,并简述随机微分方程仍然存在的诸多问题和巨大的应用前景. 


















