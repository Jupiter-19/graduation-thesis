% !TeX encoding = UTF-8
\chapter{预备知识}\label{chap2}
本章介绍随机分析的基础知识,这些知识辅助我们去理解相较于确定性的偏微分方程,随机微分方程的解构成一个$\sigma$代数,其中的元素为概率测度. 

\section{随机过程:域流、布朗运动和鞅}
设 $(\Omega,\mathcal F)$ 为可测空间,三元组 $(\Omega,\mathcal F,P)$ 是一个完备的概率空间,我们在此基础上定义随机变量.
\begin{definition}[随机变量,Random variables]
	给定两个可测空间 $(\Omega,\mathcal F)$ 和 $(\Omega^\prime,\mathcal F^\prime)$,一个映射 $X:\Omega \to \Omega^\prime$ 是 $\mathcal F/\mathcal F^\prime $-可测的,如果 $\mathcal F^\prime$-可测集的原像为 $\mathcal F$-可测集,即
	\[
	A^\prime \in {\mathcal F}^{\prime} {\Rightarrow} \{ \omega : X(\omega) \in A^ \prime \} \in \mathcal F.
	\]
\end{definition}

随机过程是一族取值在可测集 $\mathcal X$ 上的随机变量 $X = (X_t)_{t\in \mathbb T}$,由时间 $t$ 为指标. 时间参数 $t$ 涉及到一个动态的观点:在建模未来的事件时,时间越长,则事件 $\Omega$ 的不确定性累计并扩散,变得越来越不确定. 而随着时间的流失,人们获得越来越多的信息,未来某个时间点的不确定性越来越少. 

为了细致刻画不同时间的信息量,我们引入了一族递增的子$\sigma$代数 $\{\mathcal F_t\}_{t\in \mathbb T}$,称为域流. 
\begin{definition}[域流,Filtration]  
	给定完备概率空间 $(\Omega,\mathcal F,P)$,称$\mathcal F$ 上的递增$\sigma$代数 $\mathbb F = (\mathcal F_t)_{t \in \mathbb T}$ 为域流,若其满足 $\mathcal F_s \subset \mathcal F_t$, $\forall~0\le s \le t$. 
\end{definition}

$\mathcal F_t$ 可以解释为在 $t$ 时刻的信息,其随时间的推移增加. 我们想知道在 $\mathcal F_t$ 中观察到的事件首次发生时间 $\tau(\omega)$ 是否仅与观察到的信息流 $\mathcal F_t$ 有关. 这就导出停时的概念. 
\begin{definition}[停时,Stopping time]
	随机变量 $\tau: \Omega \to [0,\infty]$ 称为停时,若对于所有 $t\in \mathbb T$
	\[
	\{\tau \le t\} := \{ \omega \in \Omega : \tau(\omega) \le t \} \in \mathcal F_t.
	\]
\end{definition}	

取$\mathbb T$为离散时间,常见的随机过程有 Poisson 过程、Markov链等. 对于连续的时间区间,随机过程的基本例子为布朗运动. 
\begin{definition}[标准布朗运动,Standard Brownian motion]
	$\mathbb T$上的标准 $m$ 维布朗运动为 $\R^m$ 上的连续随机过程 $(B_t)_{t\in \mathbb T} = (B_t^1,\cdots,B_t^m)_{t\in \mathbb T}$ 满足:
	\begin{enumerate}
		\item 初始值:$B_0=0$.
		\item 独立增量:$\forall ~0\le s < t$,$B(t)-B(s)$ 与 $B(s)$ 相互独立. 
		\item 平稳增量:$\forall ~0\le s < t$,$B(t)-B(s)$ 与 $B(t-s)$ 同分布. 
		\item 正态分布:$\forall ~t>0$,$B(t) \sim N(0,I_m~t)$.
	\end{enumerate}
\end{definition}

进一步地,我们考虑连续时间鞅的定义.

\begin{definition}[鞅,Martingale]
	假设 $X=(X(t))_{t\in \mathbb T}$ 是一族随机变量,$\{\mathcal F_t\}_{t\in \mathbb T}$ 是一族单调不减的子$\sigma$代数. 如果下列条件满足:
	\begin{enumerate}
		\item 对每个 $t\ge0$,$E|X(t)|<\infty$.
		\item 对每个 $t\ge0$,$X(t) \in \mathcal F_t$.
		\item $\forall ~0\le s < t$,
		\[
			E(X(t) | \mathcal F_s) =X(s),\qquad \mathrm{a.s.}		
		\]
	\end{enumerate}
	称 $(X(t) , \mathcal F_t)_{t\in \mathbb T}$ 是鞅. 
\end{definition}

关于鞅,还有很多可以继续展开的点,如:半鞅、局部鞅、可料过程、Doob-Meyer分解,见文献 \cite{book1}.



\section{不等式}
随机分析中的误差估计,往往要涉及二次变差和求期望,因此难度很大,在介绍常用的不等式之前,首先引入符号:
\[
	\left<M, N\right>_{t}=\lim _{n \rightarrow+\infty} \sum_{i=1}^{k_{n}}\left(M_{t_{i}^{n}}-M_{t_{i-1}^{n}}\right)\left(N_{t_{i}^{n}}-N_{t_{i-1}^{n}}\right)
\]
不难得到对应的极化恒等式:
\[
	\left< M,N \right> = \frac12 \bigl( \left< M+N,M+N\right> - \left< M,M\right> -  \left< N,N\right>  \bigl).
\]
这里我们将 $\left<M,M\right>$ 简单记作 $\left<M\right>$. 例如,对于 $m$ 维 Brown 运动 $B = (B^1,\cdots,B^m)$,我们有
\[
	\left< B^i, B^j \right>_t = \delta_{ij} t.
\]
基于该符号,引入随机积分下的柯西不等式:
\begin{proposition}[Kunita-Watanabe不等式\cite{book1}]
	令 $M,N$ 为连续局部鞅,$\alpha,\beta$ 是 $\mathbb T\times \Omega$ 上的积$\sigma$代数 $\mathcal B(\mathbb T) \otimes \mathcal F$ 上的可测过程,则对任意 $t\in\mathbb  T$,
	\[
	\int_{0}^{t}\left|\alpha_{s}\right|\left|\beta_{s}\right| d~|\left<M, N\right>|_{s} \leq\left(\int_{0}^{t} \alpha_{s}^{2} d\left<M, M\right>_{s}\right)^{\frac{1}{2}}\left(\int_{0}^{t} \beta_{s}^{2} d\left<N, N\right>_{s}\right)^{\frac{1}{2}} \qquad \mathrm{a.s.}
	\]
\end{proposition}

在对误差上界估计时,以下两个不等式也非常常见. 
\begin{proposition}[Burkholder-Davis-Gundy不等式\cite{book1}]
	对任意 $p>0$,存在正常数 $c_p$ 和 $C_p$,使得对任意连续局部鞅 $M=(M_t)_{t\in\mathbb T}$ 和停时 $\tau$,有
	\[
	c_{p} E\left[\left<M\right>_{\tau}^{p / 2}\right] \leq E\left[\sup _{0 \leq t < \tau}\left|M_{t}\right|\right]^{p} \leq C_{p} E\left[\left<M\right>_{\tau}^{p / 2}\right]
	\]
\end{proposition}


%\begin{proposition}[Doob不等式]
%	对于任意停时 $\tau$ 和非负鞅或半鞅过程 $(X_t)_{t\in \mathbb T}$,有
%	\[
%	\begin{aligned}
%	&P\left[\sup _{0 \leq t \leq \tau}\left|X_{t}\right| \geq \lambda\right]  \leq \frac{E\left|X_{\tau}\right|}{\lambda}, \quad \forall \lambda>0 \\
%	&E\left[\sup _{0 \leq t \leq \tau}\left|X_{t}\right|\right]^{p}  \leq\left(\frac{p}{p-1}\right)^{p} E\left[\left|X_{\tau}\right|^{p}\right], \quad \forall p>1
%	\end{aligned}
%	\]
%\end{proposition}


在确定性的偏微分方程的误差估计中,经常要使用到 Gronwall 不等式,而 SDE 的误差估计中,同样要频繁地使用到它
\begin{lemma}[Gronwall不等式\cite{SEE}]
	令 $g$ 是 $\R_+$ 上的连续正函数,且有
	\[
	g(t) \le h(t) + C\int_0^t g(s) \md (s), \qquad 0 \le t \le T,
	\]
	则有估计式:
	\[
	g(t) \le h(t) + C\int_0^t h(s) e^{C(t-s)} \md s,\qquad 0 \le t \le T.
	\]
\end{lemma}

\section{随机积分}
随机积分的出发点是构造半鞅过程的微积分,由于其常规意义下的微分和积分不存在.为了克服无界变差引起的 Riemann-Stieltjes 积分不收敛的问题\cite{book_suiji},文献中出现多种方式定义积分 $\displaystyle\int_0^t f(s) \md B_s$. \ito 和 Stratonovich 是其中最为重要的两种方式. 

假设 $\bm{f} = (f(t),0\leq t \le T)$ 是非随机有界变差函数,那可以利用分部积分的方式定义随机积分
\[
	\int_0^t f(s) \md B(s) = B(t)f(t) - \int_0^t B(s) \md f(s),
\]
这里的 $\displaystyle\int_0^t B(s) \md f(s)$ 是 Brown 运动关于 $f$ 的 Riemann-Stieltjes 积分. 现在考虑带随机项的 $\bm{f}$,对随机过程 $\bm{f}=(f(t),0\leq t \le T)$ 做如下假设:
\begin{enumerate}
	\item[(H1)] 可测性:$f(\omega,t) : \Omega \times [0,T] \mapsto \R$ 关于 $\mathcal F_T \times \mB([0,T])$ 是可测的,即对 $\R$ 上的任意 Borel 集合 $B$,有
		\[
		\{(\omega,t) : f(\omega,t) \in B\}  \in \mathcal F_T \times \mB([0,T])];
		\] 
	\item[(H2)] 适应性:$\bm{f}$ 关于 Brown 运动 $B$ 是适应过程,即对每个 $t$ 和 $\R$ 上的任意 Borel 集 $B$,有
		\[
		\{\omega : f(\omega,t) \in B\}  \in \mathcal F_t;
		\] 
	\item[(H3)] 可积性:$\bm{f}$ 是二阶矩过程,且
		\[
		\int_0^T E f(t)^2 \md t < \infty.
		\]
\end{enumerate}

\begin{lemma} 
	令
	\[
	\mH = \{ f(\omega,t): \text{满足~(H1)\ --\ (H3)} \},
	\]
	并定义
	\[
	\left< f,g \right>_\mH = \int_0^T E(f(t)g(t)) \md t,\qquad \forall ~ f,g, \in \mH.
	\]
	那么 $( \mH,\left< \cdot,\cdot \right> )_\mH$ 是 Hilbert 空间. 进而,定义 $\| f \| _\mH = \left< f,f \right >_\mH$,那么 $(\mH,\|\cdot\|_\mH)$ 是 Banach 空间.  
\end{lemma}

类似于 Lebesgue 积分的定义方式,我们构造对应的初等随机过程. 称 $\bm{f} = (f(t) , 0 \leq t \leq T)$ 为初等随机过程,若存在 $0=t_0\le t_1 < t_2 < \cdots < t_m = T$, 以及随机变量 $\xi_0,\xi_1,\cdots,\xi_{m-1}$ 使得 $\xi_i \in \mF _{t_i}, E\xi_i^2 < \infty$ 并且
\[
	f(t) = \sum_{i=0}^{m-1} \xi_i ~ \bm{1}_{(t_i,t_{i+1}]} . 
\]

令 $\mH_0$ 为所有初等随机过程所组成的空间,假设 $f \in \mH_0$,定义 
\[
I(f) = \sum_{i=0}^{m-1} \xi_i (B(t_{i+1}) - B(t_i)).
\]
这边引导出的是 \ito 型积分,而如果设置为区间的右端点,即将 $\xi_i$ 改为 $\xi_{i+1}$,则引导向 Stratonovich 积分. 

\begin{lemma}
	$I(f)$ 具有下列性质:
	\begin{enumerate}
		\item 可加性:对任意 $f,g \in \mH_0$,
			\[ I(f+g)=I(f)+I(g); \]
		\item 零均值:对任意 $f\in \mH_0$,
			\[ EI(f) = 0;  \]
		\item 等距性质:对任意 $f\in \mH_0$,
			\[ E(I(f))^2 = \int_0^T E(f(t))^2 \md t. \]
	\end{enumerate}
\end{lemma}
\begin{proof}
	仅证明3. 由于 $\xi_i\in \mF_{t_i}$,所以与 $B(t_{i+1}) - B(t_i)$ 独立. 并且,对于 $i \neq j$
	\[
		E[ \xi_i (B(t_{i+1} - B(t_i)) \xi_j (B(t_{j+1} - B(t_j)) ] = 0. 
	\]
	于是,有
	\[
	\begin{aligned}
	E(I(f))^2 &=  E \left[ \sum_{i=0}^{m-1} \xi_i (B(t_{i+1} - B(t_i)) \right] ^2 
		 =\sum_{i=0}^{m-1} E \left[ \xi_i^2 (B(t_{i+1} - B(t_i))^2 \right] \\
		&=\sum_{i=0}^{m-1} E \xi_i^2 E(B(t_{i+1} - B(t_i))^2 
		 =\sum_{i=0}^{m-1} E \xi_i^2 (t_{i+i} - t_i) = \int_0^T E(f(t))^2 \md t.
	\end{aligned}
	\]
\end{proof}
等距性质在处理随机积分时非常重要,有
\[ \|I(f)\|_2 = \|f\|_\mH \]
通过它,将 $I(f)$ 拓展到 $\mH$ 上. 下面引理表明 $\overline{\mH_0} = \mH$.
\begin{lemma}
	任给 $f\in \mH$,存在一列 $f_n \in \mH_0$ 使得当 $n\to \infty$ 时,
	\[ \| f_n-f\|_\mH \to 0. \]
\end{lemma}
因而存在随机变量 $I(f)$,使得
	\begin{equation} \label{eq1.2.1}
		\lim_{n\to\infty} \|I(f_n) - I(f) \|_\mH = 0. 
	\end{equation}
称由(\ref{eq1.2.1})定义的 $I(f)$ 为 $f$ 关于Brown运动的积分. $I(f)$ 与序列 $\{f_n\}$ 的选择无关,并且在 $\mH$ 中具有可加性、零均值、等距性质. 由此,随机积分有如下推论:
\begin{cor}
	令 $f:\R \to \R$ 是非随机连续函数,那么对任意给定的 $t\ge 0$,
	\begin{enumerate}
		\item $\displaystyle \int_0^T f(B_s) \md B_s = \lim_{n\to\infty} \sum_{i=1}^n f(B_{t_{i-1}}) (B_{t_i} - B_{t_{i-1}})$;
		\item $\displaystyle \int_0^T f(s) \md B_s  = \lim_{n\to\infty} \sum_{i=1}^n f(t_i^*) (B_{t_i} - B_{t_{i-1}})$,\quad $t_i^* \in  [ t_{i-},t_i ] $.
	\end{enumerate}
\end{cor}

通过划分、取点、求和、取极限的方式计算 \ito 积分非常麻烦, 因此引入下面的重要公式. 

\begin{lemma}[\ito 公式\cite{book_suiji}]
	\[
	\begin{aligned}
	f\left(t, X_{t}\right)=f\left(0, X_{0}\right)&+\int_{0}^{t} \frac{\partial f}{\partial t}\left(u, X_{u}\right) \md u+\sum_{i=1}^{d} \int_{0}^{t} \frac{\partial f}{\partial x_{i}}\left(u, X_{u}\right) \md X_{u}^{i} \\
	&+\frac{1}{2} \sum_{i, j=1}^{d} \int_{0}^{t} \frac{\partial^{2} f}{\partial x_{i} \partial x_{j}}\left(u, X_{u}\right) \md \left<X^{i}, X^{j}\right>_{u}
	\end{aligned}
	\]
\end{lemma}

与微积分学中的 Newton-Leibniz 公式比较,\ito 公式出现二次变差项,这是Brown运动无界变差引起的. 


\section{随机微分方程}
固定概率空间 $(\Omega,\mF,\mathbb F=(\mF_t)_{t\in T},P)$ 和 $\mathbb F$ 对应的 $m$ 维Brown运动 $B = (B^1,\cdots,B^m)$. 给定函数 $b(t,x,\omega) = (b_i(t,x,\omega))_{1 \le i \le m}$ 和 $\sigma(x,t,\omega) = (\sigma_{ij}(t,x,\omega))_{1\le i\le d , 1\le j \le m}$. 且对任意 $\omega$,函数 $b(\cdot,\cdot,\omega)$ 和 $\sigma(\cdot,\cdot,\omega)$ 是 $[0,T]\times \R^d$ 上的 Borel 可测函数,我们考虑的如下的随机微分方程(SDE)(\ref{SDDE})和随机控制问题(SCP)(\ref{SCP}):
\begin{equation}\label{SDDE}
	\md X_t = b(t,X_t) \md t + \sigma(t,X_t) \md B_t;
\end{equation}
\begin{equation}\label{SCP}
	\md X_t = b(t,X_t,\omega) \md t + \sigma(t,X_t,\omega) \md B_t.\
\end{equation}
我们的文章主要关心 (\ref{SDDE}) 的数值解法,对于特定类型的 (\ref{SCP}) 问题,也会简单介绍其求解方法. 

因为随机微分方程求解的困难,在理论研究的过程中,我们会将问题(\ref{SDDE})进一步简化为自治随机常微分方程
\begin{equation}\label{SODE}
	%\left\{\begin{array}{l}
	\md X(t) = b(X(t)) \md t+\sigma(X(t)) \md B_{t}, \quad 0<t \leq T 
	%X(0)=X_{0}
	%\end{array}\right.
\end{equation}
这类自治系统可以通过转移半群来表述. 


\begin{remark}
	在一些文献\cite{SG1,SG2,ref_parareal}的论述中,使用 Stratonovich 意义的SDE:
	\[
	\md X_t = f(t,X_t) \md t + g(t,X_t) \circ \md B_t,
	\]
	其用 \ito 方式表述,为
	\[
	\md X = \left(f(t,X_t) + \frac12 D g(t,X_t) g(t,X_t)\right) \md t + g(t,X_t) \md B_t.
	\]
\end{remark}


\begin{definition}[SDE的强解]
	初始状态为 $t$ 的向量可测过程 $X = (X^1,\cdots,X^d)$ 的 SDE(\ref{SDDE}) 若满足:
	\[
	\int_{t}^{s}\left|b\left(u, X_{u}\right)\right| \md u+\int_{t}^{s}\left|\sigma\left(u, X_{u}\right)\right|^{2} \md u<\infty, \quad a . s ., \forall t \leq s \in T. 
	\]
	则强解满足如下关系式:
	\[
	X_s = X_t + \int_t^s b(u,X_u) \md u + \int_t^s \sigma(u,X_u) \md B_u,\qquad t\le s \le T.
	\]
	或者满足如下的分量关系式:
	\[
	X_{s}^{i}=X_{t}^{i}+\int_{t}^{s} b_{i}\left(u, X_{u}\right) \md u+\sum_{j=1}^{d} \int_{t}^{s} \sigma_{i j}\left(u, X_{u}\right) \md B_{u}^{j}, \quad t \leq s \in \mathbb{T}, 1 \leq i \leq d
	\]
\end{definition}
注意到 SDE 的强解是一个连续的随机过程. SDE问题(\ref{SDDE})需要满足相应的 Lipschitz条件和增长条件,才能保证强解的存在唯一性. 

\begin{assump*}[A]
	存在常数 $K$ 和实值随机过程 $\kappa$,使得对任意 $t\in [0,T],\quad \omega \in \Omega, x,y,\in \R^n$,有
	\begin{equation}\label{assumeA-1}
		\begin{aligned}
			|b(t,x,\omega) - b(t,y,\omega)|+|\sigma(t,x,\omega)-\sigma(t,y,\omega)| \le & K|x-y| \\
			|b(t,x,\omega)|+|\sigma(t,x,\omega)| \le & \kappa_t(\omega) + K|x|,
		\end{aligned}
	\end{equation}
	且
	\begin{equation}\label{assumeA-2}
		E \left[ \int_0^T |\kappa_u|^2 \md u \right] < \infty ,\qquad \forall  ~ t\in [0,T].
	\end{equation}
\end{assump*}

\begin{theorem}[强解的存在唯一性]
	当满足条件(\ref{assumeA-1})和(\ref{assumeA-2})时,SDE(\ref{SDDE})在任意时间 $t$ 存在强解. 进一步地,对任意 $\mF_t$ 可测随机变量 $\xi$,若存在某个 $p>1$,使得 $E|\xi|^p < \infty$,则对从时间 $t$ 开始的SDE,即 $X_t=\xi$,所以强解都是不可区分的,即对于任意两个强解 $X$ 和 $Y$,$P[X_s=Y_s,\forall~ t\le s < T]=1$. 同时,该强解还是 $p$方可积的:对任意 $t<T$,存在常数 $C_{_T}$,使得
	\[
	E\left[\sup _{t \leq s \leq T}\left|X_{s}\right|^{p}\right] \leq C_{_T}\left(1+E\left[|\xi|^{p}\right]\right)
	\]
\end{theorem}

该结果是随机SDE中的标准结论,证明可以见 Gihman 的著作\cite{book_ref_GI}、Ikeda 的著作\cite{book_ref_NI}、Krylo的著作\cite{book_ref_NVK}或Protter的著作\cite{book_ref_PP}
SDE中另一个重要的结论是对 Fokker-Planck 方程研究而导出的Feynman-Kac公式:


\iffalse 
\begin{definition}[SDE的弱解]
	
\end{definition}

\begin{theorem}[弱解的存在唯一性]
	
\end{theorem}
\fi



\begin{theorem}[Feynman-Kac表示]
	令 $v\in C^{1,2}([0,T]\times \R^d) \cap C^0([0,T] \times \R^d)$ 关于 $x$ 的导数有界,则柯西问题的解存在. 且 $v$ 存在如下表示:
	\[
	v(t, x)=E\left[\int_{t}^{T} e^{-\int_{t}^{s} r\left(u, X_{u}^{t, x}\right) \md u} b\left(s, X_{s}^{t, x}\right) \md s+e^{-\int_{t}^{T} r\left(u, X_{u}^{t, x}\right) \md u} \sigma \left(X_{T}^{t, x}\right)\right]
	\]
	$\forall ~ (t,x) \in [0,T] \times \R^d$.
\end{theorem}









