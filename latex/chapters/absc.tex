%% 中文摘要
\chapter*{\centerline{摘\quad 要}}
\chaptermark{摘要}
\addcontentsline{toc}{chapter}{摘要}

\vspace{1em}

本文研究带守恒量的自治随机微分方程的数值解,分别从单条样本轨道的解和整体的概率解的角度进行分析. 
本文首先介绍随机偏微分方程的理论基础和经典的数值算法,包括 Euler-Maruyama 格式、Milstein 格式和 Taylor 格式,这部分的讨论与确定性偏微分方程是很不一样的,这是因为随机偏微分方程的解具有概率意义,且每条样本轨道是无界变差函数. 

对于长时间演化的问题,本文分析了一维情况在满足一定条件时,存在唯一的稳定解,并给出了高效的加速算法. 对于带守恒量的问题,本文介绍离散梯度格式和投射方法,这两种方法均能在保证误差精度的前提下保持守恒量不变. 值得注意的是,离散梯度格式仅能处理带单一守恒量的问题,而投射方法可以处理带多个守恒量的问题. 

本文对投射方法做了改进,原先论文的投射方向选取为垂直方向,并不具有唯一性,导致迭代点可能出现不在守恒量对应的流形上的情况. 本文的投射方法选取为距离最近的方向,保证了迭代点的守恒量不变,且使用拉格朗日乘数法,操作方便快捷. 

对于带守恒量的问题,本文注意到自治随机微分方程可能存在多个稳定解,且稳定解与守恒量相对应. 结合离散梯度格式和投射方法,本文提出了对稳定解不唯一的自治随机微分方程的稳定解的加速算法. 

在最后的数值实验部分,本文验证经典数值格式的收敛阶和稳定解的加速算法. 并探究带单一守恒量的二维随机 Kubo oscillator 方程和带多个守恒量的三维随机 Lotka-Volterra 方程. 通过一系列的数值实验加深对这两个随机系统的解的认识. 

\vspace{1em}

\noindent{\bfseries\Fangsong 关键词:}~~自治随机微分方程;Milstein格式;概率解;守恒量;加速算法;



